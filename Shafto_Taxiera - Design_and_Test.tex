\documentclass[a4paper]{article}

%% Language and font encodings
\usepackage[english]{babel}
\usepackage[utf8x]{inputenc}
\usepackage[T1]{fontenc}
\usepackage{outlines}

%% Sets page size and margins
\usepackage[a4paper,top=3cm,bottom=2cm,left=3cm,right=3cm,marginparwidth=1.75cm]{geometry}

%% Useful packages
\usepackage{graphicx}
\usepackage[colorinlistoftodos]{todonotes}
\usepackage[colorlinks=true, allcolors=blue]{hyperref}

\title{FAT System Design}
\author{Robin Shafto, Alex Taxiera}

\begin{document}
\maketitle

\section{Shell Description}

The shell will read and parse user input, if "exit" was input it exits, otherwise it executes the command given and loops for more input.
\section{New Functions}

%A list of all new functions.  State the function prototype, what each input parameter is and what the return value is.  Describe the operations to be performed by the function

\subsection{mapFree}
Prototype:\hspace{0.2cm}char* mapFree()\\
Input:\hspace{0.88cm}None\\
Returns:\hspace{0.51cm}A free-space bitmap\\\\
This function creates a free-space bitmap to track the free space.

\subsection{updateBitmap}
Prototype:\hspace{0.2cm}bool updateBitmap(char* operation, int bits)\\
Input:\hspace{0.88cm}operation is whether space was added or removed, and bits are how many bits were altered.\\
Returns:\hspace{0.51cm}True if successful\\\\
This function will update the bitmap with what space is free.

\subsection{getCurrentDirectoryContents}
Prototype:\hspace{0.2cm}string getCurrentDirectoryContents()\\
Input:\hspace{0.88cm}None\\
Returns:\hspace{0.51cm}An array of strings containing all files and directories in current directory.\\\\
This function will check all the contents of the current directory, providing a reference for commands like cd, cat, and touch.

\subsection{getDirectoryContents}
Prototype:\hspace{0.2cm}string getDirectoryContents(char* directory)\\
Input:\hspace{0.88cm}Directory to read\\
Returns:\hspace{0.51cm}An array of strings containing all files and directories in directory\\\\
This function will check all the contents of the given directory, it will call isDirectory first.

\subsection{fileExists}
Prototype:\hspace{0.2cm}bool fileExists(char* fileName)\\
Input:\hspace{0.88cm}File that you wish to check.\\
Returns:\hspace{0.51cm}True if the file exists.\\\\
This function checks the existence of a file by calling getCurrentDirectoryContents and comparing.

\subsection{isDirectory}
Prototype:\hspace{0.2cm}bool isDirectory(char* fileName)\\
Input:\hspace{0.88cm}File that you wish to check.\\
Returns:\hspace{0.51cm}True if the file is a directory.\\\\
This function checks the type of a file by calling fileExists.

\subsection{printFile}
Prototype:\hspace{0.2cm}void printFile(string fileName)\\
Input:\hspace{0.88cm}Name of file to print contents\\
Returns:\hspace{0.51cm}Nothing\\\\
This function will print the contents of a file. 

\subsection{error}
Prototype:\hspace{0.2cm}void error(int errorNumber)\\
Input:\hspace{0.88cm}Number of error to print.\\
Returns:\hspace{0.51cm}Nothing\\\\
This function will be sent an integer value when there should be an error thrown and will then print the corresponding string.

\subsection{getFileSize}
Prototype:\hspace{0.2cm}int getFileSize(char *filename)\\
Input:\hspace{0.88cm}Name of file to get size.\\
Returns:\hspace{0.51cm}Size of file.\\\\
This function will find the size of a file.

\subsection{checkRange}
Prototype:\hspace{0.2cm}bool checkRange(int x, int y)\\
Input:\hspace{0.88cm}Range of FAT entries to check.\\
Returns:\hspace{0.51cm}True if valid inputs, otherwise false.\\\\
This function verifies range when looking to read FAT entries.

\subsection{readFAT12Table}
Prototype:\hspace{0.2cm}char* readFAT12Table(char* buffer)\\
Input:\hspace{0.88cm}Which FAT table to read.\\
Returns:\hspace{0.51cm}A buffer with the desire FAT table.\\\\
This function reads a FAT table for us.

\subsection{freeSector}
Prototype:\hspace{0.2cm}bool freeSector(int sectorNumber)\\
Input:\hspace{0.88cm}The data sector of the target file.\\
Returns:\hspace{0.51cm}True if it was successful.\\\\
This function frees sectors.

\subsection{findFree}
Prototype:\hspace{0.2cm}int findFree(char* directory)\\
Input:\hspace{0.88cm}Find free entries in the directory\\
Returns:\hspace{0.51cm}The free entries\\\\
This function finds free entries.

\subsection{allocateSector}
Prototype:\hspace{0.2cm}bool allocateSector(char* directory)\\
Input:\hspace{0.88cm}Allocate another sector to the directory\\
Returns:\hspace{0.51cm}True if successful\\\\
This function allocates more sectors for file creation.

\section{Command Processes}

%For each command other than pbs and pfe, describe (in steps) how the command execution will be performed using the functions you described above.  

\subsection{cat \textit{x}}
\begin{enumerate}
\item Call isDirectory
\item If the file exists and it is not a directory, call printFile
\item Else return an error
\end{enumerate}

\subsection{cd \textit{x}}
\begin{enumerate}
\item Call isDirectory
\item If the directory exists then move to it
\item Else return an error
\end{enumerate}

\subsection{df}
\begin{enumerate}
\item Print the free-space bitmap.
\end{enumerate}

\subsection{ls \textit{x}}
\begin{enumerate}
\item If more than one argument return error
\item If no arguments call getCurrentDirectoryContents
\item Print directory contents
\item If one argument call isDirectory
\item If true call getDirectoryContents
\item Print directory contents
\item If false call fileExists
\item If true display file details
\item If false return error
\end{enumerate}

\subsection{mkdir \textit{x}}
\begin{enumerate}
\item If more than one argument return error
\item Call isDirectory
\item If true return error
\item If false call findFree
\item If no free entries call allocateSector
\item If false return error
\item If true create the directory and call updateBitmap
\end{enumerate}

\subsection{pwd}
\begin{enumerate}
\item Print absolute path to the current directory
\end{enumerate}

\subsection{rm \textit{x}}
\begin{enumerate}
\item If more than one argument return error
\item Call fileExists
\item If false return error
\item Call isDirectory
\item If true return an error
\item if false remove the file
\item Call freeSector and updateBitmap
\end{enumerate}

\subsection{rmdir \textit{x}}
\begin{enumerate}
\item If more than one argument return error
\item Call fileExists
\item If false return error
\item Call isDirectory
\item If false return error
\item If true call getDirectoryContents
\item If array is not empty return error
\item If array is empty then remove entry
\item Call freeSector and updateBitmap
\end{enumerate}

\subsection{touch \textit{x}}
\begin{enumerate}
\item Command should ignore all arguments except first one
\item Call fileExists
\item If true return error
\item If false call findFree
\item If no free entries call allocateSector
\item If false return error
\item If true create the file and call updateBitmap
\end{enumerate}

\section{Test Plans}

The shell must be completed first so that the other components may be tested. It will be tested for compatibility with the provided functions, then with stubs of the commands, then with the supplementary functions and working commands.

% In order to make sure that your system is working properly, you must implement a test plan.  Your team will develop a systematic procedure to test your system.  A test plan should have detail information on what is the expected output for a given input for all commands.   As you are testing your system, you know where a problem may occur and you will have to fix it.  An exhaustive test plan will catch any features that you might have missed in your design phase and save you many hours of re-writing code, during the implementation phase.   At the end, your program will have to pass all your test cases to be able to claim that it is well tested. These test cases are important, thus your team must create as many critical test cases as possible.  Then, it will help you to make sure your system is robust.  If you change your test plan, make sure you have your plan updated because you must turn in your final test plan documentation.

\subsection{Test Cases}

\subsubsection{cat \textit{x}}
\begin{enumerate}
\item {\textbf{For a file: } prints file contents}
\item {\textbf{For a directory: } prints error message}
\item {\textbf{For incorrect number of arguments: } prints error message}
\item {\textbf{For unspecified: } prints error message}
\end{enumerate}

\subsubsection{cd \textit{x}}
\begin{enumerate}
\item {\textbf{For a specified directory: } changes to specified directory}
\item {\textbf{For unspecified: } changes to home directory}
\end{enumerate}

\subsubsection{df}
\begin{enumerate}
\item {\textbf{For all inputs: } creates free-space bitmap mapping each cluster to a bit}
\end{enumerate}

\subsubsection{ls \textit{x}}
\begin{enumerate}
\item {\textbf{For a filename: } lists name, extension, type, FLC, and size of file}
\item {\textbf{For directory: } lists names of entries with their extensions,  types, FLC, and sizes as well as the current and parent directories}
\item {\textbf{For unspecified: } for the current directory, lists names of entries with their extensions,  types, FLC, and sizes as well as the current and parent directories}
\item {\textbf{For two or more arguments: } returns error message}
\end{enumerate}

\subsubsection{mkdir \textit{x}}
\begin{enumerate}
\item {\textbf{For existing directory: } prints that x already exists}
\item {\textbf{For insufficient memory: } print appropriate message}
\item {\textbf{For a number of arguments not equal to one: } returns error message}
\item {\textbf{For a new directory x in an existing target directory with space: } creates x in target directory}
\end{enumerate}

\subsubsection{pwd}
\begin{enumerate}
\item {\textbf{For all cases: } prints path to current directory}
\end{enumerate}

\subsubsection{rm \textit{x}}
\begin{enumerate}
\item {\textbf{For nonexistent file: } prints that file does not exist}
\item {\textbf{For a directory: } print error message}
\item {\textbf{For a number of arguments not equal to one: } prints error message}
\item {\textbf{For a file x in a directory: } removes x from parent directory, optimizes parent directory storage, frees data sectors of target file, updates data structures}
\end{enumerate}

\subsubsection{rmdir \textit{x}}
\begin{enumerate}
\item {\textbf{For nonexistent directory: } prints that file does not exist}
\item {\textbf{For a file: } print error message}
\item {\textbf{For a number of arguments not equal to one: } prints error message}
\item {\textbf{For non-empty directory: } prints error message}
\item {\textbf{For a directory x: } removes x from parent directory, optimizes parent directory storage, frees data sectors of target file, updates data structures}
\end{enumerate}

\subsubsection{touch \textit{x}}
\begin{enumerate}
\item {\textbf{For existing file x: } prints that x already exists}
\item {\textbf{For insufficient memory: } print appropriate message}
\item {\textbf{For a number of arguments not equal to one: } returns error message}
\item {\textbf{For a new file x, given sufficient space: } creates x in target directory}
\end{enumerate}

\end{document}